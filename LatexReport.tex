\documentclass{article}
\usepackage{amsmath}
\begin{document}

%\begin{titlepage}
%	\begin{center}
%	\line(1,0){300} \\
%	[2mm]
%	\huge{\bfseries Generation Economics, Assignment 2} \\
%	[2mm]
%	\line(1,0){200}\\
%	[1,5cm]
%	\textsc{\LARGE Bram Poldervaart} \\
%	[0.75cm]
%	\textsc{\Large Assignment 2 }\\ 
%	[9cm]
%	\end{center}
%	\begin{flushright}
%	\textsc{\large Bram Poldervaart \\
%	ANR \\
%	\# 785939 \\
%	March 15, 2016 \\}
%	\end{flushright}
%	
%\end{titlepage}
\title{Generational Economics - Assignment 2}
\author{Bram Poldervaart}
\date{March 15, 2016}
\maketitle

\section{Exercise 1}\label {sec:ex1}
Assume a Diamond model as described in the reader where the consumer has preferences $U_t = c^y_t*c^o_t+1$, there is no technological progress and production is described by $y_t = k^\alpha$.
\\
\\
i. Compute the capital-labour ratio in the steady state. What is the effect of an increase in $\alpha$ on the steady-state capital-labour ratio?\\
\line(1,0){300} \\
\begin{eqnarray} 
k_{t+1}= \frac{s_t }{1+g}  \nonumber \\
\nonumber \\ 
f(k_t)=y_t = k^\alpha_t \mbox{ and } g=0 \nonumber \\
\nonumber \\
k_{t+1} = \frac {1}{2+p} (k^\alpha_t - k_t\alpha k_t^{\alpha-1}) \nonumber 
\end{eqnarray}

\textit{To complete the steady state we use the difference equation: $k_{t+1} = k_t = k$.} \\

\begin{eqnarray}
k_{t+1} = k_t = \frac {(1-\alpha)k_t^\alpha}{2+p} \nonumber \\
k^{1-\alpha} = \frac{1-\alpha}{2+p} \nonumber \\
k= (\frac{1-\alpha}{2+p})^\frac{1}{1-\alpha} \nonumber & \mbox{ is the steady state.}
\end{eqnarray} 

\textit{An increase in $\alpha$ will let k decrease and convert to 0. That means that in the steady state the capital-labour ration would move to an increase in labour relative to capital.}
 \\
 \\
 \\
 \\
ii. Suppose the government introduces a tax ? on interest income. All revenues of this tax are transferred to the elderly in the same period as a pension benefit. So the budget constraints become:
\begin{equation}
c_t^y=w-s_t
\end{equation}
\begin{equation}
c_{t+1}^o = (1+(1-\varsigma)r)s_t+T
\end{equation}
Note that the level of the pension benefit T is given for the individual and cannot be affected by saving more or less.\\
Derive the Euler equation. \\
\\
\\
\textit{First, we write down the utility function and the production function:}\\
$U_t = C_t^yC_{t+1}^o$\\
\\ 
\textit{We combine this with the consumption function at t for the young and t+1 for old, we get:}\\
$U_t = (w-s_t)((1+(1-\varsigma)r)s_t+T)$\\
\\
\textit{If we derive this equation:}

$\frac{\partial U}{\partial s_t} = 0 $\\
$\frac{\partial U}{\partial s_t} = w(1+(1-\varsigma)r) - 2s_t(1+(1-\varsigma)r) = 0$\\
\\
$s_t = \frac{w(1+(1-\varsigma)r)-T}{2(1+(1-\varsigma)r)} $\\
\\
\textit{By substituting $s_t$ in both $C_{t+1}^o$ and $C_t^y$, we can derive the Euler Equation:}\\
$\frac{C_{t+1}^o}{C_t^y} = \frac{w(1+(1-\varsigma)r)+T}{\frac{w(1+(1-\varsigma)r)}{1+(1-\varsigma)r}+T} =1+(1-\varsigma)r$



\newpage
\section{Excersise 2}\label {sec:ex2}
a. Assume that production is described by the following neoclassical production function without labour-augmenting technological progress: $Y = F (K, L)$\\
Derive the dynamic equation for capital per person (k) assuming a constant savings rate.\\
\\
\textit{Since $\Delta$ is equal to difference between the savings and the effective depreciation and we assume there is no labour-augmenting technology (g=0), we find:}\\
$\Delta k(t) = \frac{\Delta K(t)}{L(t)} - \frac{K(t)}{L(t)} \frac{\Delta L(t)}{L(t)} 
= \frac{sY(t)-\delta K(t)}{L(t)} - K(t)n$\\
\\
$\Delta k(t) = sf(k(t)) - (\delta +n)k(t)$\\
\\
\\
b. Assume a Solow-Swan model as described in the reader, with a Cobb-Douglas production function $Y = 1.5K^{0.5}L^{0.5}$ and $x = 0.5$, $n = 0.5$, $\delta = 0.25$, $\sigma = 0.5$.\\
\\
i. Compute the capital-labour ration $k^*$ in the steady state.\\
\\
$(1+g)=(1+n)(1+\alpha) = 1.5 * 1.5 = 2.25 \Leftrightarrow g=1.25$\\
$f(k_t)=AK^{0.5}L^{-0.5} = Ak^{0.5} = 1.5k^{0.5}$\\
\\
\textit{The steady state can then by calculated:} $i = (\delta + g)k$ where $i=\sigma f(k) = 0.75k^{0.5}$
$0.75k^{0.5} = 1.5k \Leftrightarrow k_{ss}^*=0.25$\\
\\
\\
ii. Is the steady state dynamically efficient or dynamically inefficient?\\
\\
\textit{The Golden Rule (GR) can be derived using the following equation:}\\
$f'(k) = \delta +g \Leftrightarrow 0.75k^{-0.5}=1.5\Leftrightarrow k_{GR}=0.25$ \\
\\
$k_{SS}=k_{GR}\Rightarrow$ \textit{Neither dynamically efficient nor dynamically inefficient.}






	
\end{document}